
\documentclass[a4paper,14pt]{extarticle}

\usepackage{cmap}

\usepackage[T2A]{fontenc}
\usepackage[utf8x]{inputenc}
\usepackage[russian,english]{babel}

\usepackage[a4paper,margin=1.5cm,footskip=1cm,left=2cm,right=1.5cm,top=1.5cm
	,bottom=2.0cm]{geometry}
\usepackage{enumitem}
\usepackage{textcase}

\setlist[description]{leftmargin=\parindent,labelindent=\parindent}

\begin{document}

\setcounter{secnumdepth}{0}

\begin{titlepage}

	\begin{center}
		\MakeTextUppercase{ministry of education and science of the russian
			federation}
		\bigbreak
		\MakeTextUppercase{federal state-funded educational institution of higher
			education novosibirsk state technical university}
		
		\vspace{125pt}
		
		Foreign Languages Department of Engineering Faculties - 2 \\
		\vspace{100pt}
		\textbf{Summary on article} \\
		\smallbreak
		Kalman Filtering in R
		\vspace{75pt}
	\end{center}

	\begin{flushleft}
	\begin{tabbing}
	Master student:\quad\quad \= K. K. Gorbunov \\
	\\
	Faculty:          \> Applied Mathematics and Computer Science \\
	Department:       \> Theoretical and Applied Computer Science \\
	\\
	Major:            \> Applied Mathematics and Computer Science \\
	\\
	Research adviser: \> Assoc. Prof. O. S. Chernikova, Ph.D. (Eng.)
	\\
	Teacher of English: \> O. S. Atamanova
	\end{tabbing}
	\end{flushleft}

	\begin{center}
		\vspace{\fill}
		Novosibirsk, 2016
	\end{center}

\end{titlepage}

\newpage

TODO: add table of contents

\newpage

\begin{center}
	Kalman Kiltering in R
\end{center}

\textit{Keywords:} state space models, Kalman filter, time series, R. \\

In the article <<Kalman Filtering in R>> by Fernando Tussel, University of the
Basque Country, Kalman filtering algorithms and implementations are described,
discussed and overviewed, in particular, implementations in R programming
language. \\

On the one hand, Kalman filter is not very complex algorithm which can be
implemented by hand rather quickly. But the more precise, stable and robust
implementation one needs, the more effort one must take to implement the
algorithm that would workaround different caveats. That is stated in the
introduction to the article. \\

The 2nd section formally describes the object model and Kalman filter
equations as well as algorithm different additional features to workaround
some caveats or provide some more advanced additional functionality. \\

In the 3rd section particular R packages containing
algorithms implementation based on Kalman filter are
described and overviewed. \\

The 4th section summarizes algorithms' particular features and provides their speed
comparison, running them on different datasets built in R. \\

The 5th section provides summary discussion on the algorithms \ldots \\

\newpage

\begin{center}
	Glossary
\end{center}

\end{document}
# vim: ts=2 sw=2
