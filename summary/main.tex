
\documentclass[a4paper,14pt]{extarticle}

\usepackage{cmap}

\usepackage[T2A]{fontenc}
\usepackage[utf8x]{inputenc}
\usepackage[russian,english]{babel}

\usepackage[a4paper,margin=1.5cm,footskip=1cm,left=2cm,right=1.5cm,top=1.5cm
	,bottom=2.0cm]{geometry}
\usepackage{enumitem}
\usepackage{textcase}

\setlist[description]{leftmargin=\parindent,labelindent=\parindent}

\begin{document}

\setcounter{secnumdepth}{0}

\begin{titlepage}

	\begin{center}
		\MakeTextUppercase{ministry of education and science of the russian
			federation}
		\bigbreak
		\MakeTextUppercase{federal state-funded educational institution of higher
			education novosibirsk state technical university}
		
		\vspace{125pt}
		
		Foreign Languages Department of Engineering Faculties - 2 \\
		\vspace{100pt}
		\textbf{Summary on article} \\
		\smallbreak
		Kalman Filtering in R
		\vspace{75pt}
	\end{center}

	\begin{flushleft}
	\begin{tabbing}
	Master student:\quad\quad \= K. K. Gorbunov \\
	\\
	Faculty:          \> Applied Mathematics and Computer Science \\
	Department:       \> Theoretical and Applied Computer Science \\
	\\
	Major:            \> Applied Mathematics and Computer Science \\
	\\
	Research adviser: \> Assoc. Prof. O. S. Chernikova, Ph.D. (Eng.)
	\\
	Teacher of English: \> O. S. Atamanova
	\end{tabbing}
	\end{flushleft}

	\begin{center}
		\vspace{\fill}
		Novosibirsk, 2016
	\end{center}

\end{titlepage}

\newpage

\begin{center}
	Kalman Kiltering in R
\end{center}

\textit{Keywords:} state space models, Kalman filter, time series, R. \\

In the article <<Kalman Filtering in R>> issued in the Journal of Statistical
Software in March of 2011 written by Fernando Tussel, University of the
Basque Country, Kalman filtering algorithms and their implementations are
described, discussed and overviewed, in particular, implementations in R
programming language. The algorithms can solve state and parameter estimation
problem. \\

On the one hand, Kalman filter is not very complex algorithm which can be
implemented by hand rather quickly. But the more precise, stable and robust
implementation one needs, the more effort one must take to implement the
algorithm that would workaround some different caveats. That is stated in the
introduction to the article. \\

The 2nd section formally describes general state-space object model and Kalman
filter equations as well as algorithm's different additional features to
workaround some caveats or provide some more advanced additional
functionality. \\

First, information filter implemetation of the Kalman filter
can allow one to specify complete uncertainty about the initial value of a 
component of the state vector, although this implementation in general
requires more computational effort. Using covariance filter one must set 
large covariance values to specify complete uncertainty, but this can lead
to large rounding errors. Another way dealing with this caveat is using
exact diffuse initialization, an option described below. \\

One more caveat is numerical instability of the straightforward Kalman filter
implementation that can be resolved by using "Joseph stabilized form" of the
covariance update equation or using square root filter implementation. \\

In the case when measurement matrix is diagonal ob block-diagonal
sequential processing feature can be used to decrease computational efforts and
maybe even increase numerical stability, especially in combination with square
root filtering feature, but in this case the computational advantages seem
unclear. \\

Smoothing is the feature function to estimate the state given all past and
future measurements. In some cases, and notable for the Bayesian analysis of
the state space model, it is of interest to generate random samples of state
and disturbance vectors, conditional on the observations. \\

Exact diffuse initial conditions is the technique of specifying and dealing
with the complete uncertainties of initial conditions described by Durbin,
and Koopman, Grewal and Andrews. \\

Maximum likelihood estimation of the state-space model is the algorithm to
find unknown parameters present in the model in different matrices in different
combinations. This is a working horse of systems identification. An alternative
can be so called EM algorithm. \\

In the 3rd section particular R packages containing
algorithms implementation based on Kalman filter are
described and overviewed, including their programming interfaces.
These packages are dse, sspir, dlm, FKF, KFAS, and
some other briefly mentioned. \\

The 4th section summarizes algorithms' particular features and provides their
speed and accuracy comparison, running them different datasets such as Nile
data, exchange rates data and a regression artificially generated data. \\

The 5th section provides summary discussion on the overviewed algorithms
overview.

\begin{center}
	References
\end{center}

Anderson BDO, Moore JB (1979). Optimal Filtering. Prentice-Hall. \\

Anderson E, Bai Z, Bischof C, Blackford S, Demmel J, Dongarra J, Croz JD, Greenbaum \\

A, Hammarling S, McKenney A, Sorensen D (1999). LAPACK Users’ Guide. 3rd edition.
SIAM. \\

Bierman GJ (1977). Factorization Methods for Discrete Sequential Estimation.
		Dover Publications. \\

Bucy RS, Joseph PD (1968). Filtering for Stochastic Processes with Application to Guidance.
Interscience, New York. \\

Byrd RH, Lu P, Nocedal J, Zhu C (1995). “A Limited Memory Algorithm for Bound
Constrained Optimization.” SIAM Journal on Scientific Computing, 16, 1190–1208.
Cameletti M (2009). Stem: Spatio-Temporal Models in R. R package version 1.0, URL
http://CRAN.R-project.org/package=Stem. \\

Carter CK, Kohn R (1994). “On Gibbs Sampling for State Space Models.” Biometrika, 81(3),
541–553. \\

Chambers JM (2008). Software for Data Analysis: Programming with R. Springer-Verlag,
New York. \\

Chambers JM, Hastie TJ (1992). Statistical Models in S. Wadsworth \& Brooks/Cole, Pacific
Grove. \\

Cowpertwait PSP, Metcalfe AV (2009). Introductory Time Series with R. Springer-Verlag,
New York. \\

de Jong P (1995). “The Simulation Smoother for Time Series Models.” Biometrika, 82(2),
339–350. \\

Dethlefsen C, Lundbye-Christensen S (2006). “Formulating State Space Models in R with
Focus on Longitudinal Regression Models.” Journal of Statistical Software, 16(1), 1–15.
URL http://www.jstatsoft.org/v16/i01/. \\

Dethlefsen C, Lundbye-Christensen S, Christensen AL (2009). sspir: State Space Models in
R. R package version 0.2.8, URL http://CRAN.R-project.org/pa \\

\newpage

\end{document}
# vim: ts=2 sw=2
