\documentclass[12pt,a4paper]{article}

\usepackage[utf8]{inputenc}
\usepackage[russian,english]{babel}

\begin{document}

GEOPHYSICAL RESEARCH LETTERS, VOL. 17, NO. 3, PAGES 199-202, MARCH 1990

АЛГОРИТМ АВТОМАТИЧЕСКОЙ КОРРЕКЦИИ ДАННЫХ GPS

Жоффрей Бльюитт

Jet Propulsion Laboratory, California Institute of Technology, Pasadena
Калифорнийский Технологический Институт, Пасадена

\emph{Аннотация.} Алгоритм, разработанный для автоматической коррекции данных,
Глобальной Системы Позиционирования, а именно: удаления аномальных значений, 
идентификации и коррекции скачков фазы --- то, что не зависит от нестабильности
часов, селективной доступности, кинематики приемник-спутник и тропосферных
условий. Этот алгоритм, названный ''TurboEdit'', предназначен для двухчастотных
измерений фазы несущей в абсолютном режиме и требует (1) доступность P-кодовых 
дальномерных измерений и (2) плавное изменение содержания электронов в 
ионосфере. Последнее требование может не выполняться строго, если в 
используемом для анализа данных программном обеспечении заложены методики 
разрешения неоднозначности фазы для оценивания скачков фазы. ''TurboEdit''
был протестирован на большой выборке данных, полученных во время реализации
проекта ''CASA UNO'', эти данные содержали свыше 2500 скачков фазы.
Вмешатальство аналитиков потребовалось лишь в 1 \% от общего числа проходов
спутника над базовой станцией, почти все эти случаи были обусловлены
трудностями в экстраполировании дисперсии ионосферной задержки. В настоящее
время алгоритм адаптируют для коррекции данных в реальном времени в приемнике
''Rogue'' для целей непрерывного мониторинга.

Введение.

Данные измерений фазы несущей сигнала Глобальной Системы Позиционирования
используются в течение нескольких последних лет для измерений региональных
геодезических сетей с субсантиметровой точностью. Результаты использования
програмнной системы ''GIPSY'' показали согласие 


\end{document}
