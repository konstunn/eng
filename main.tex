\documentclass[12pt,a4paper]{article}

\usepackage[T2A]{fontenc}
\usepackage[utf8]{inputenc}
\usepackage[russian,english]{babel}

\usepackage[a4paper,margin=1.5cm,footskip=1cm,left=2cm,right=1.5cm,top=1.5cm
	,bottom=2.0cm]{geometry}

\usepackage{amsmath}

\begin{document}

\begin{center}
GEOPHYSICAL RESEARCH LETTERS, VOL. 17, NO. 3, PAGES 199-202, MARCH 1990
\bigbreak
АЛГОРИТМ АВТОМАТИЧЕСКОЙ КОРРЕКЦИИ ДАННЫХ GPS
\medbreak
Жоффрей Бльюитт
\medbreak
\mbox{Лаборатория Реактивного Движения, Калифорнийский Технологический Институт, Пасадена}
\end{center}

\emph{Аннотация.} Разработан алгоритм для автоматической коррекции данных,
Глобальной Системы Позиционирования, а именно: удаления аномальных значений, 
идентификации и коррекции скачков фазы --- то, что не зависит от нестабильности
часов, селективной доступности, кинематики приемник-спутник и тропосферных
условий. Этот алгоритм, названный ''TurboEdit'', предназначен для двухчастотных
измерений фазы несущей в абсолютном режиме и требует (1) доступность P-кодовых 
дальномерных измерений и (2) плавное изменение содержания электронов в 
ионосфере. Последнее требование может не выполняться строго, если в 
используемом для анализа данных программном обеспечении заложены методики 
разрешения неоднозначности фазы для оценивания скачков фазы. ''TurboEdit''
был протестирован на большой выборке данных, полученных во время реализации
проекта ''CASA UNO'', эти данные содержали свыше 2500 скачков фазы.
Вмешатальство аналитиков потребовалось лишь в 1 \% от общего числа проходов
спутника над базовой станцией, почти все эти случаи были обусловлены
трудностями в экстраполировании дисперсии ионосферной задержки. В настоящее
время алгоритм адаптируют для коррекции данных в реальном времени в приемнике
''Rogue'' для целей непрерывного мониторинга.

\medbreak
\begin{center}
	Введение. 
\end{center}

\medbreak
Данные измерений фазы несущей сигнала Глобальной Системы Позиционирования
используются в течение нескольких последних лет для измерений региональных
геодезических сетей с субсантиметровой точностью [e.g., Dong and Bock, 1989]. 
Результаты использования програмнной системы ''GIPSY'' хорошо согласуются в 
горизонтальных составляющих базовых линий с результатами интерферометрии, 
проведенной на очень длинной базовой линии в Калифорнии [Blewitt, 1989].
Относительная погрешность решения трехмерной задачи не превышала 15 миллиардных
частей для расстояния 2000 км [Lichten and Bertiger, 1989]. Условием достижения
такой высокой точности в решениях задач геодезии на основе GPS является наличие
алгоритмов надежного обнаружения и коррекции (где это возможно) скачков 
неоднозначностей фазы,
порожденных потерей приемником захвата GPS-сигнала. Без должной коррекции
дальнейшая обработка GPS-данных с целью получения геофизических данных
бесполезна.

Для обработки данных GPS в общем случае применяют различные эвристические
методы, большинство из которых работают с разностными наблюдениями между парами
станций и парами спутников, таким образом уменьшая инструментальные погрешности.
Случаи, когда алгоритмы отрабатывали неудачно, анализировались экспертами
с использованием интерактивных средств визуализации с последующей ''ручной''
коррекцией. Такая потребность в довольно интенсивном вмешательстве в
автоматизированный процесс обработки данных была главной преградой на пути к
повышению эффективности, устойчивости и воспроизводительности анализа и
обработки данных. В массиве данных, полученных в ходе
эксперимента ''CASA UNO'' было обнаружено более 2500 скачков фазы, тогда стало
понятно, что необходимо минимизировать участие специалистов-аналитиков, так как
иначе потребовалось бы большое число людей, каждый из которых так или иначе
использовал бы свои техники и подходы к анализу и обработке, нарушая
единообразие и воспроизводимость.

В данной работе представлен надежный алгоритм автоматической коррекции
двухчастотных измерений фаз несущей GPS-сигнала, поступающих с приемников,
оснащенных возможностью проводить измерения псевдодальности по P-коду. Данная
методика особенно интересна тем, что в отличие от большинства алгоритмов,
не использует дифференциальные измерения. Следовательно, данный алгоритм может
использоваться в реальном времени приемниками при проведении полевых работ.
К тому же, не требуется моделирование задержки сигнала на радиотрассе, поэтому
алгоритм применим для задач кинематики (однако, как будет показано, должны быть
учтены некоторые антенные эффекты).

Алгоритм был включен в программный пакет ''GIPSY'' в качестве модуля под
названием ''TurboEdit''. Далее будет описан принцип работы алгоритма,
приведен анализ его быстродействия, будут рассмотрены возможности адаптации
алгоритма для приемников, не оснащенными возможностями проведения кодовых
измерений, а также влияние ионосферных условий на изменение задержки сигнала
для фазовых измерений.

\begin{center} Опредения и описание модели наблюдений \end{center}

Многие идеи, описанные здесь, были разработаны в работах Бльюитт (Blewitt)
[1989] для целей разрешения неоднозначности фаз несущей, которое заключается
в определении целого числа длин волн, связанного с первым измерением фазы в
сеансе. С последующими фазовыми измерениями будет связано то же целое число,
при условии, что сохраняется захват сигнала приемником. Потери захвата порождают
скачки фазы, которые в данной работе называются ''проскальзывание фазы''.
(В качестве предупреждения заметим, что некоторые инженеры не используют этот
термин, если скачок связан с зашумлением сигнала --- с низким отношением
сигнал-шум. В данной работе для удобства такого различия не проводится.)
Временной ряд данных фазовых измерений, в котором нет скачков фазы, называется
''фазосвязной траекторией''. Целью коррекции GPS-данных является (1) исключение
аномальных наблюдений, (2) идентификация скачков фазы где это возможно и (4)
инициировать параметр неоднозначности фазы для каждого сеанса с последующим его
уточнением, оцениванием, используя уже отлаженные для этого методики [Blewitt,
1989; Dong and Bock, 1989]. Рассмотрим следующую модель наблюдений фаз несущей
и кодовых псевдодальностей для некоторой пары приемник-спутник (то есть
абсолютные, недифференциальные наблюдения)

\begin{equation}
	L_1 \equiv -c \Phi_1 / f_1 = \rho - I f_2^2 / (f_1^2 - f_2^2) + \lambda_1
	b_1
\end{equation}

\begin{equation}
	L_2 \equiv -c \Phi_2 / f_2 = \rho - I f_1^2 / (f_1^2 - f_2^2) + \lambda_2
	b_2
\end{equation}

\begin{equation}
	P_1 = \rho + I f_2^2 / (f_1^2 - f_2^2)
\end{equation}

\begin{equation}
	P_2 = \rho + I f_1^2 / (f_1^2 - f_2^2)
\end{equation}

где $\Phi_1$ и $\Phi_2$ --- измеренные фазы несущей, $L_1$ и $L_2$ --- фазы
несущей, выраженные в дальностях, $P_1$ и $P_2$ --- кодовые псевдодальности,
$c$ --- скорость света, частоты несущей $f_1 = 1.54 \times 10.23$ МГц и
$f_2 = 120 \times 10.23$ МГц, и длины волн $\lambda_1 \approx 19.0$ см и
$\lambda_2 \approx 24.4$ см. Член $\rho$ включает в себя геометрическую
дальность, тропосферную задержку, поправки часов, селективная доступность
(быстрые изменения опорной частоты 10.23 МГц) и любые другие эффекты которые
симметрично, одинаково влияют на все данные. Член $I$ --- ионосферная задержка;
$b_1$ и $b_2$ --- целочисленные случайные величины, описывающие скачки фазы.

\end{document}
