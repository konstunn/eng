\documentclass[a4paper, 14pt]{extarticle}

\usepackage[a4paper,margin=1.5cm,footskip=1cm,left=2cm,right=1.5cm,top=1.5cm
	,bottom=2.0cm]{geometry}


\begin{document}

Alright. Good day! Thank you for coming. I'm gonna give a talk about the role
of English language worldwide with an emphasis on why English is also or maybe
even especially important for so called techmakers. I wonder if you know who
techmakers are? (pause) - Techmakers are all kinds of engineers, scientists,
technical guys who just make things around us work. Like
most of you in this audience, technical university students,
researchers, engineers, teaching staff. 

My name is Konstantin Gorbunov. I am a first year master of science program
student. You see the title 'English, Techmaker, Do You Speak It?' and mentioned
these two nice guys with guns. Do you know who these guys are? - Well, obviously,
they are not techmakers. 

---

They are characters of Quentin Tarantino's action movie Pulp Fiction. You know,
the title of my talk is a remade quotation from this movie. If you haven't
watched it yet, go ahead and watch it today, in English.

---

Okay, so what's the plan?

Here you can see a short plan briefly describing the reasonal points on why
English is important.

You know, it's not only bare business for me to make you think so, but
I have a personal reason to think so by myself and persuade you to think so.
To think that English is a must.

The history also matters. There I'll mention a historical reason why English is
so widely spread.

And the next point we're gonna discuss is why we need English to help us
carry out our research and development.

And in conlusion I will try to give you even more motivation and provide some
tips for improving your English.

---

So, let's get started.

You know, I got my bachelors degree in Instrumentation at the faculty of
Automation and Computer Engineering, dealing a lot with hardware development
and design.

Now I am working on my masters' degree at the faculty Applied Math and Computer
Science dealing more with data and algorithms.

I'm telling you this not to show off, to say that I'm kind a super genius guy.
No!

The point is that it does not matter what you are dealing with: hardware,
software, maths, algorithms. Doesn't matter. The matter is that you will
nevertheless need English and take much benefit from it if you learn it.

And I can't remember a discipline that I didn't get benefit from using my
knowledge of English.

That's my personal experience, personal reason why I think that English is a
must.

---

The history also matters. 

---

As it was said by Martin Luther King, Jr. We are not
history makers, we are made by history.

So to speak going to some historical reason going back to the times of British
Empire taking large world territory we've got what we've got. The situation is
the following.

---

There are approx. at least 600 million people around the world speaking English.

--- 

English has become the language of world diplomacy, sport, science,
technology, education, the Internet, mass media and tourism and so forth.
Certainly we are mostly interested in science, technology and education. Cause
we are techmakers.

---

That is why the most terrific point for me and I'm sure that it's also so for
most of you, English is the language of software development. Now, dear
colleages, I want you answer the only one question. Have you ever seen a
programming language used in production which keywords was from Japanese,
Chinese or German? - I have not.

---

So to continue. I've also decided to draw your attention to the site stack
exchange . com. It's a great place of community-driven forums with topics from
statistics to languages.  And even cooking. And even alsmost whatever you want.
I suppose it might be interesting for you to visit it.  

---

Okay, let's go the next point. Speaking and understanding English speech and
text is about general literacy and awareness. The more languages you know, the
more aware and literal you are, the more points of view and opinions you can
evaluate. The more information you can deal with. That's important.

---

Cause now we live in the Age of Infomation. They say that the one who owns
information, that one rules the world. With development of the Internet the
more information became accessible. But not all information is useful, however.
Actually, that one who can extract knowledge from information that one rules
the world. And to extract knowledge one needs deep understanding of data he
deals with. Assuming you are dealing with data in English - the better you know
English, the more useful information you can extract.

---

In my conclusion I want to provide you some tips on improving your English.
Some of them might look for you obvious, though. Watch movies, listen to music,
read, speak and surf the web in English.

Learn English, and those guys will never come for you.

Thank you for your attention. Now I want you to feel free to ask any questions.
Say your what. If you feel shy, you can e-mail me or find me
at vk.com with ease. The information provided here is more than enough - very
useful information, isn't it?

\end{document}
