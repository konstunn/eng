%% LyX 2.2.1 created this file.  For more info, see http://www.lyx.org/.
%% Do not edit unless you really know what you are doing.
\documentclass[10pt]{beamer}\usepackage[]{graphicx}\usepackage[]{color}
%% maxwidth is the original width if it is less than linewidth
%% otherwise use linewidth (to make sure the graphics do not exceed the margin)
\makeatletter
\def\maxwidth{ %
  \ifdim\Gin@nat@width>\linewidth
    \linewidth
  \else
    \Gin@nat@width
  \fi
}
\makeatother

\definecolor{fgcolor}{rgb}{0.345, 0.345, 0.345}
\newcommand{\hlnum}[1]{\textcolor[rgb]{0.686,0.059,0.569}{#1}}%
\newcommand{\hlstr}[1]{\textcolor[rgb]{0.192,0.494,0.8}{#1}}%
\newcommand{\hlcom}[1]{\textcolor[rgb]{0.678,0.584,0.686}{\textit{#1}}}%
\newcommand{\hlopt}[1]{\textcolor[rgb]{0,0,0}{#1}}%
\newcommand{\hlstd}[1]{\textcolor[rgb]{0.345,0.345,0.345}{#1}}%
\newcommand{\hlkwa}[1]{\textcolor[rgb]{0.161,0.373,0.58}{\textbf{#1}}}%
\newcommand{\hlkwb}[1]{\textcolor[rgb]{0.69,0.353,0.396}{#1}}%
\newcommand{\hlkwc}[1]{\textcolor[rgb]{0.333,0.667,0.333}{#1}}%
\newcommand{\hlkwd}[1]{\textcolor[rgb]{0.737,0.353,0.396}{\textbf{#1}}}%
\let\hlipl\hlkwb

\usepackage{framed}
\makeatletter
\newenvironment{kframe}{%
 \def\at@end@of@kframe{}%
 \ifinner\ifhmode%
  \def\at@end@of@kframe{\end{minipage}}%
  \begin{minipage}{\columnwidth}%
 \fi\fi%
 \def\FrameCommand##1{\hskip\@totalleftmargin \hskip-\fboxsep
 \colorbox{shadecolor}{##1}\hskip-\fboxsep
     % There is no \\@totalrightmargin, so:
     \hskip-\linewidth \hskip-\@totalleftmargin \hskip\columnwidth}%
 \MakeFramed {\advance\hsize-\width
   \@totalleftmargin\z@ \linewidth\hsize
   \@setminipage}}%
 {\par\unskip\endMakeFramed%
 \at@end@of@kframe}
\makeatother

\definecolor{shadecolor}{rgb}{.97, .97, .97}
\definecolor{messagecolor}{rgb}{0, 0, 0}
\definecolor{warningcolor}{rgb}{1, 0, 1}
\definecolor{errorcolor}{rgb}{1, 0, 0}
\newenvironment{knitrout}{}{} % an empty environment to be redefined in TeX

\usepackage{alltt}
\usepackage[T1]{fontenc}
\setcounter{secnumdepth}{3}
\setcounter{tocdepth}{3}
\usepackage{url}
\ifx\hypersetup\undefined
  \AtBeginDocument{%
    \hypersetup{unicode=true,pdfusetitle,
 bookmarks=true,bookmarksnumbered=false,bookmarksopen=false,
 breaklinks=false,pdfborder={0 0 0},pdfborderstyle={},backref=false,colorlinks=false}
  }
\else
  \hypersetup{unicode=true,pdfusetitle,
 bookmarks=true,bookmarksnumbered=false,bookmarksopen=false,
 breaklinks=false,pdfborder={0 0 0},pdfborderstyle={},backref=false,colorlinks=false}
\fi
\usepackage{breakurl}

\makeatletter

%%%%%%%%%%%%%%%%%%%%%%%%%%%%%% LyX specific LaTeX commands.
\providecommand{\LyX}{\texorpdfstring%
  {L\kern-.1667em\lower.25em\hbox{Y}\kern-.125emX\@}
  {LyX}}

%%%%%%%%%%%%%%%%%%%%%%%%%%%%%% Textclass specific LaTeX commands.
 % this default might be overridden by plain title style
 \newcommand\makebeamertitle{\frame{\maketitle}}%
 % (ERT) argument for the TOC
 \AtBeginDocument{%
   \let\origtableofcontents=\tableofcontents
   \def\tableofcontents{\@ifnextchar[{\origtableofcontents}{\gobbletableofcontents}}
   \def\gobbletableofcontents#1{\origtableofcontents}
 }

%%%%%%%%%%%%%%%%%%%%%%%%%%%%%% User specified LaTeX commands.
\usetheme{PaloAlto}

\makeatother
\IfFileExists{upquote.sty}{\usepackage{upquote}}{}
\begin{document}


\title[knitR]{Dynamic Reporting and Automated Reproducible Research with knitR}

\author{Konstantin Gorbunov}
\institute[NSTU]
{
	Novosibirsk State Technical University \\
	Faculty of Applied Mathematics and Computer Science
}
\date[PTI, 2017]{Progress Through Innovations Conference\\ March $30^{th}$, 2017}

\makebeamertitle

\section{\TeX\ /\ \LaTeX}
\begin{frame}{\TeX\ /\ \LaTeX}
	\begin{itemize}
		\item Flexible
		\item Truly cross-platform
		\item Free
		\item Open Source
	\end{itemize}
\end{frame}

\section{R}
\begin{frame}{R}
	\begin{itemize}
		\item Extensible C/C++, Fortran 
		\item Large set of packages at CRAN
		\item Large open-source community of users
		\item MATLAB-like syntax and capabilities
		\item Flexible
		\item Cross-Platform
		\item Free
		\item Open Source
	\end{itemize}
\end{frame}

\section{Random numbers generating example}
\begin{frame}[fragile]{Random numbers generating example}



\begin{knitrout}\footnotesize
\definecolor{shadecolor}{rgb}{0.969, 0.969, 0.969}\color{fgcolor}\begin{kframe}
\begin{alltt}
\hlcom{# create some random numbers}
\hlstd{(x}\hlkwb{=}\hlkwd{rnorm}\hlstd{(}\hlnum{20}\hlstd{))}
\end{alltt}
\begin{verbatim}
##  [1] -1.29240163  0.24834594  1.35380188  0.30777762
##  [5]  1.31334679  0.85411975  0.67567890  1.09849915
##  [9]  0.98391119 -0.07897815 -1.46002699 -0.03174679
## [13] -0.45018527  0.86144631 -0.99112981 -0.94034910
## [17] -0.69034585  0.83782120 -0.36527831 -1.66987352
\end{verbatim}
\begin{alltt}
\hlkwd{mean}\hlstd{(x)}
\end{alltt}
\begin{verbatim}
## [1] 0.02822166
\end{verbatim}
\begin{alltt}
\hlkwd{var}\hlstd{(x)}
\end{alltt}
\begin{verbatim}
## [1] 0.9357885
\end{verbatim}
\end{kframe}
\end{knitrout}

BTW, the first element of \texttt{x} is -1.2924016. (This is possible with
the use of\texttt{ \textbackslash{}Sexpr\{\}} \LaTeX\ environment.)
\end{frame}

\section{Plotting example}
\begin{frame}[fragile]{Plotting example}

\begin{knitrout}\footnotesize
\definecolor{shadecolor}{rgb}{0.969, 0.969, 0.969}\color{fgcolor}\begin{kframe}
\begin{alltt}
\hlstd{caption} \hlkwb{<-} \hlstr{'This is the caption of the figure.'}
\hlkwd{par}\hlstd{(}\hlkwc{las}\hlstd{=}\hlnum{1}\hlstd{,}\hlkwc{mar}\hlstd{=}\hlkwd{c}\hlstd{(}\hlnum{4}\hlstd{,}\hlnum{4}\hlstd{,}\hlnum{.1}\hlstd{,}\hlnum{.1}\hlstd{))}  \hlcom{# tick labels direction}
\hlkwd{boxplot}\hlstd{(x)}
\hlkwd{hist}\hlstd{(x,}\hlkwc{main}\hlstd{=}\hlstr{''}\hlstd{,}\hlkwc{col}\hlstd{=}\hlstr{"blue"}\hlstd{,}\hlkwc{probability}\hlstd{=}\hlnum{TRUE}\hlstd{)}
\hlkwd{lines}\hlstd{(}\hlkwd{density}\hlstd{(x),}\hlkwc{col}\hlstd{=}\hlstr{"red"}\hlstd{)}
\end{alltt}
\end{kframe}\begin{figure}

{\centering \includegraphics[width=.45\linewidth]{figure/beamer-boring-plots-1} 
\includegraphics[width=.45\linewidth]{figure/beamer-boring-plots-2} 

}

\caption[This is the caption of the figure]{This is the caption of the figure.}\label{fig:boring-plots}
\end{figure}


\end{knitrout}

\end{frame}

\section{Table example}
\begin{frame}[fragile]{Table example}

\begin{kframe}
\begin{alltt}
\hlkwd{library}\hlstd{(xtable)}
\hlkwd{options}\hlstd{(}\hlkwc{xtable.floating}\hlstd{=F)}
\hlkwd{options}\hlstd{(}\hlkwc{xtable.timestamp}\hlstd{=}\hlstr{""}\hlstd{)}
\hlkwd{data}\hlstd{(tli)}
\hlkwd{xtable}\hlstd{(tli[}\hlnum{1}\hlopt{:}\hlnum{5}\hlstd{,])}
\end{alltt}
\end{kframe}% latex table generated in R 3.3.3 by xtable 1.8-2 package
% 
\begin{tabular}{rrlllr}
  \hline
 & grade & sex & disadvg & ethnicty & tlimth \\ 
  \hline
1 &   6 & M & YES & HISPANIC &  43 \\ 
  2 &   7 & M & NO & BLACK &  88 \\ 
  3 &   5 & F & YES & HISPANIC &  34 \\ 
  4 &   3 & M & YES & HISPANIC &  65 \\ 
  5 &   8 & M & YES & WHITE &  75 \\ 
   \hline
\end{tabular}


\end{frame}

\section{\LaTeX\ <- knitR <- R}
\begin{frame}{\LaTeX\ <- knitR <- R}
	\begin{itemize}
		\item Reproducible research
		\item No copy-pasting
		\item Script your research and report
	\end{itemize}
\end{frame}

\section{Contacts}
\begin{frame}{Contacts}
	\begin{itemize}
		\item E-mail: konstunn@ngs.ru
		\item GitHub: github.com/konstunn
	\end{itemize}
\end{frame}


\section{References}
\begin{frame}{References}
	\begin{thebibliography}{9}

	\bibitem{knitr16} Yihui Xie (2016). knitr: A General-Purpose Package for
		Dynamic Report Generation in R. R package version 1.15.1.

	\bibitem{r} R Core Team (2014). R: A language and environment for statistical
		computing. R Foundation for Statistical Computing, Vienna, Austria.
		URL http://www.R-project.org/.

	\bibitem{Knuth:90} D.~E. Knuth, {\em The {\TeX}book}.
		Addison\,\textendash\,Wesley, 1990.

	\bibitem{Lamport:94}
		L.~Lamport, {\em \LaTeX: A Document Preparation System}.
		Addison\,\textendash\,Wesley, 1994.

	\end{thebibliography}

\end{frame}


\end{document}
